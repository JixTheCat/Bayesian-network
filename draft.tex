

\title{Measuring environmental sustainability with Australian vine growers using Bayesian Networks}

\section{Methods}

In agriculture, the consideration of environment is critical to the success of the industry. The question of environmental sustainability has continued to grow in importance as the industry is affected by international market demands, disease pressures, natural disasters and social change \citep{wineaustraliaNationalVintageReport2022,wineaustraliaNationalVintageReport2020,wineaustraliaNationalVintageReport2021,cassonMultidisciplinaryApproachAssess2022}. The question of sustainability requires the consideration of a multitude of factors, not only due to physical differences between locations/habitats but also due to priorities driven by a myriad of perspectives from as consumers, ecosystems, government and industry \citep{baianoOverviewSustainabilityWine2021,wayeCarbonFootprintsFood2008}. It is important to consider each factor with respect to its relevant perspectives, and to be explicit in how each of these factors and perspectives are defined \citep{santiago-brownSustainabilityAssessmentWineGrape2015}.

Collaboration between experts from different perspectives is a useful method to reconcile opinions of the significance of different factors and their contribution to sustainable outcomes \citep{dichiaraCollaborativeApproachAchieving2024}. In Australia, the views of both industry and consumer are united in their priority to improve greenhouse gas emissions and resource inputs through approaches such as regulating chemicals, water use and fertilisers \citep{dumbrellComparingAustralianPublic2024}. While there is a general agreement on broader regulatory needs, the industry requirements vary greatly between different wine regions; due to climate, soil types and water resources \citep{abbalDecisionSupportSystem2016,agostaRegionalClimateVariability2012}.

Methods for measuring sustainability comprise a wide variety of approaches with the use of index systems being common \citep{gehringerMappingSustainabilityMeasurement2024}. Other approaches utilise the identification of key factors, and their measurements as a proxy to indicate impact to, or the health of, the environment \citep{floresWhatSustainabilityWine2018}. The measuring of sustainability through the use of these tools can help to manage risk and promote resilience for industries through identifying sustainable goals and key factors of interest that contribute to positive outcomes \citep{klemesAssessingMeasuringEnvironmental2015, beckerSustainabilityScienceManaging2024}. Within this study we focus on measuring key factors and establishing how they interact, and ultimately effect the surrounding environment; the chosen method for this study is a Bayesian Network.

\subsection{Bayesian Networks}

A Bayesian Network is a graphical model, being comprised of nodes and edges. Uniquely, Bayesian Networks contain the network's operations within the nodes themselves instead of across the edges. This is done in the form of probability tables, with connections being informed by the edges. There is simplicity within the calculations done within a node, as it is fundamentally achieved through sum and division operations. However, great complexity is introduced when a node is informed by many parents, which can also have many parents themselves. The change in a single parent node effects all children of the network; often resulting in intractable systems, especially when nodes may have numerous classes \citep{korbBayesianArtificialIntelligence2011}. The strength of this complexity over simple operations, is the ability to encapsulate incredibly complex systems with simple definitions.

The use of a Bayesian Network allows for the ability to include perspectives from industry experts and create a causal structure. The graphical nature of the Bayesian Network also helps to illustrate factors within the model and communicate the complexity and relationships of factors to one another; and ultimately a factor's relation direct or indirect, to a vineyard's environmental impact \citep{pourretBayesianNetworksPractical2008}. Furthermore, data collected across Australia could also be used to help inform both scenarios and the model alongside expert opinion.

The calculations of a Bayesian Network are based on Bayes' theorem, shown as follows
\begin{equation}\label{bayesformula}
P(A|B)= {P(B|A)P(A)}\over{P(B)}
\end{equation}.
Where we calculate the probability of an event A given an event B, as demonstrated by \cite{bayesLIIEssaySolving1763}. Within a Bayesian Network, an event can be considered any given node and its outcome one of the node's classes. Events are connected by edges within the network, with each node containing the probability of an outcome given parent nodes' outcomes. This is an iterative application of \ref{bayesformula}, which can be achieved through

\begin{equation}
        \mathbb(P) (X_{1}=x_{1},...X_{n}=x_{n})=\prod_{n}^{i=1}p(x_{i}|x_{Parents(i)})
\end{equation},
for a selected node $i$ given the parents and their local conditional distributions.

The iterative nature, or the ability to show the difference in probable outcomes of different sets of events and their uncertainty allows for the ability to conduct inference. This type of Bayesian inference shows the difference between the likelihood of various outcomes with and without different evidence of prior nodes outcomes being present. The sets of different outcomes then stored as a joint probability distribution, which can be decomposed into the conditional probabilities of each set of possible events occurring and their likelihoods.

One of the greatest abilities within this style of inference is Marginalization. In which, the outcome of a particular event can be determined irrespective of other outcomes. This type of marginalisation is intuitively represented as nodes conditional independence is illustrated inherently through the structure of the graph by its edges.

Bayesian Networks created in this study were created using BayesFusion \citep{bayesfusionGeNIeModelerUSER2022} to calculate the networks and their probabilities. The use of Bayes fuse software was used both through the GUI interface and the SMILE API \citep{bayesfusionGeNIeModelerUSER2022} system using C++ \citep{ISO:2012:III}.

\subsection{Elicitation}

The Bayesian Network was constructed using a mixture of data and expert elicitation. The format followed recommendations from \cite{korbBayesianArtificialIntelligence2011,pitchforthProposedValidationFramework2013}. Where, we progressed iteratively through from problem definition, to structure, to parametarisation, and then to validation. Experts were initially informed regarding the process to establish an understanding of what each step would be comprised of, as well as highlighting common pitfalls and how we would address them. With common problems being listed by both authors being:
\begin{enumerate}
        \item Not understanding the problem context
        \item complexity without value
        \item Getting arc directions wrong
        \item too many parent nodes
        \item Bias expert estimates
        \item Inconsistent probabilities
        \item Being absolutely certain
\end{enumerate}
To address these issues, experts were kept well-informed of them, as well as variable definitions and problem scope. Open communication was maintained during the entire process to help answer questions regarding the model, process, and to maintain transparency. Communication included in-person and electronic one-on-ones and group sessions; with elicitation of expert knowledge being primarily conducted through a series of workshops with the panel of industry experts.

Each workshop had a primary focus such as: scope, structure or parameterisation; with, prior concepts being built upon iteratively. The first important establishment was the scope of the network, and how well it could or couldn't be adapted to a wide variety of circumstances within the Australian winegrowing industry. Following this was the discussion around the structure of the network, which entailed the description of relationships and stricter definitions of each factor to one another. Finally, variables were paramatised using conditional probabilities.

To introduce new concepts and how they interacted, preliminary simplified versions (strawman examples) were provided to experts; with the intention of these examples to be examined, critiqued, and improved upon. The use of these strawman examples enabled a more engaged and effective review process. Experts were initially able to interact with elements of the model and see how they worked together. The strawman examples were crafted from a foundation of literature and data to offer a basic, layman's understanding of key factors and concepts, serving as a starting point for further development and discussion. 

\subsection{CPT Interpolating}

As is often the case, the most arduous endeavour is to attribute probabilities to events defined within the Bayesian Network \citep{korbBayesianArtificialIntelligence2011}. Due to the growing complexity of the network throughout the study each event was described by binary classes. With the classes depicting a positive or negative contribution to a vineyard's environmental outcome. Furthermore, the complexity of the variables, particularly water use, were overwhelmingly interrelated to operations across the vineyard. To attempt to capture this complexity but not dilute the ability to work with various scenarios the conditional probabilities were interpolated.

This method utilised interpolation between two anchored points. One point represented the case of all factors being true/positive and the other represented all factors being false/negative. The interpolation was achieved by designating each factor an attributed weight; where, weights were indicative of a factor's overall influence to its child node. During workshop discussion participants were the most comfortable using scores out of 10 to describe a factor's influence on its child node.

Using this method, any given probability can be calculated using
\begin{equation}
        p=x\sum_{i}^{n} + y\sum_{m}^{i} 
\end{equation},
for n number of true or contributing factors and m false or not contributing factors. We know the solutions for when all contributing factors are true ($p_t$) or all were false ($p_f$). Where,
\begin{equation}
        p_t=x\sum_{i}^{n} 
        \text{ and }
        p_f=y\sum_{m}^{i}
\end{equation}.
From here values for $x$ and $y$ are able to be derived using the sum of the provided weights.

\subsection{Validation}

The method for validation sought to address concerns presented by \citep{pitchforthProposedValidationFramework2013}; who listed the following:  nomological, face, content, concurrent, convergent, discriminant and predictive validity. Validation steps were partly incorporated during the process of constructing the Bayesian network itself. Specifically when considering the nomological validity of the network; the expert panel was asked directly regarding its ability to extrapolate beyond a single region, or scenario. In part, the selection process for experts, aspired to be diverse within their regional knowledge and expertise in an attempt to broaden the ability to critique choices. Although it was agreed that multiple scenarios were well captured by the trees structure, minor additions would be needed to encapsulate the differences between specific winegrowing regions. This requirement aligned with literature, which highlighted how regional variation is considered a driving factor for many vineyard outcomes and priorities \citep{abbalDecisionSupportSystem2016,agostaRegionalClimateVariability2012,soarClimateDriversRed2008}.

The Face and content validity of the network was questioned directly by the expert panel through the structure and conditional probability phases. The choice of binary discretisation was due to two reasons; that it captured the relevant contribution to the node in question and limited the model from becoming to overly complex and circumstantial. This helped to avoid some overtly specific scenarios in favour for broader vineyard decisions, whilst allowing the model to address these behaviours in the future by specifically targeting a region and incorporating its considerations. This approach went hand in hand with the content validation. Where, it was found that the limitation of the model was primarily regional, which aligned again with the findings of other researchers \citep{abbalDecisionSupportSystem2016,ellisUsingBayesianGrowth2020,agostaRegionalClimateVariability2012,barriguinhaVineyardYieldEstimation2021,brockRelationSoilOrganic2011}. Specifically, the question of water resources, which was central to the model and highly interrelated \citep{carmonaUseParticipatoryObjectOriented2011}.

Similarly, the convergent validity of the model was accomplished through the broader discretisation of nodes into the binary forms of positive and negative affects on the environmental impact of the vineyard. This manner of discretisation allowed a clear delineation in localised subnetworks within the model. The alignment of the model to other research helped to further answer questions of concurrent validation. In that, it was found that other studies emphasised the same variables selected by our experts. This was especially true regarding the consideration of yield and regional factors that influence vineyard outcomes \cite{abbalDecisionSupportSystem2016,campsGrapeHarvestYield2012,hallWithinseasonTemporalVariation2011}. Furthermore, these types of variables are highly prevalent in greater agriculture studies that emphasise general outcomes and resource use such as yield and water respectively \citep{heFruitYieldPrediction2022,laurentLocalInfluenceClimate2022}.

The discriminant validation of the model was the most difficult, as the definitions of each node changed over the course of the model's inception. This brought the difficulty that nodes connected by edges would have to be reconsidered each time a definition changed. This issue was simply overcome through explicit definitions accessible to all participants, and by directly addressing issues caused by definitions before further development of the model. Again, the simplicity of binary state nodes helped to isolate definitions more distinctly, as well as employing methods such as node splitting, removal and adding additional nodes \citep{korbBayesianArtificialIntelligence2011}.

The structure of the model itself was similar to other models in the presence of variables but differed in some of their relation due to the terminal node being environmental impact. The paramitisation of the nodes did differ to other studies, as every node was related and thought of with regard, to the terminal node and how they subsequently affected the environmental outcomes of a vineyard and not other more prevalent outcomes such as yield \citep{laurentReviewIssuesMethods2021}. The discriminant validity of the model was highlighted through the acknowledgement of the model's requirement to be tuned to any specific wine region before being able to be properly employed. Furthermore, the predictive validity was able to be well established through two methods, the use of a case study and the consideration of extreme outcomes.

\subsection{Case study: Coonawarra}

Due to the regional limitations of the model experts proposed a case study of a specific and well understood region: Coonawarra in South Australia. The choice of the region was due to the prevalence of data within and around the region, as well as experts' understanding of the region's unique challenges and benefits to winegrowing.  The Coonawarra model was built off of the original Bayesian Network's structure with site specific conditional probabilities being entered and scrutinised by the expert panel.

\section{References}

\bibliography{references} % This points to the references.bib file - the file extention is automatically added.


\end{linenumbers}
\end{document}

\endinput