
%%%%%%%%%%%%%%%%%%%%%%%%%%%%%
%       Workshop ideas      %
%%%%%%%%%%%%%%%%%%%%%%%%%%%%%

% Ice breaker
% - who are you what are you involved in.
% Who knows who?
%



%%%%%%%%%%%%%%%%%%%%%%%%%%%
% Notes for this article! %
%%%%%%%%%%%%%%%%%%%%%%%%%%%

% The forces which drive the social, economic and environmental sectors, the decision-makers who have the power or influence to accept or reject change, cannot be ignored, if change is to be achieved (van Kerkhoff and Lebel, 2006).It may be that part of the problem with the progress of sustainable development in the past has been the traditional divide between those undertaking the systematic research and those on the ground required to take up and act on the results of such. The dissemination of research results has usually only occurred on completion, in a unidirectional, linear communication (van Kerkhoff and Lebel, 2006), whereas, as grounded researchers explain, the ownership of ideas and a willingness to implement these relies on a more collaborative approach from the outset, before research has even commenced. It requires acknowledging that experts exist in all spheres and need to be able to share information and debate, in order to come to agreement on issues, assessment tools, priorities for remediation and an optimal approach to solving problems (Corbin and Strauss, 2008; Johnson and Mengersen, 2009), e.g. improving sustainability.

% It is necessary to understand the impacts of each system on one another, for example, the impact of industries on climate change, diminishing resources, economic shifts and social change. In turn, it is necessary to understand the impact of variables such as these on an industry, its future viability, its surrounding communities and the economies they feed or depend on (Brundtland, 2007). Thus, a tool is required that can measure complexity within and between systems and model how changes in one element, positive or negative, might flow on to others. Only then can strategies for sustainability be developed with reduced risk of unintended negative consequences (Johnson and Mengersen, 2012).



%%%%%%%%%%%%%%%%%%%%%%%%%%%%

% references to check:

%%%%
% A triple bottom line planning tool for measuring sustainability; A systems approach to sustinability using the Australian dairy industry as a case study.

% Capra, 1996; 
% Johnson and Mengersen, 2012)

%%%%%%%%%%%%
% Definitions:

%%
% Iterative Bayesian network developement cycle
%
% \autocite{johnsonIntegratedBayesianNetwork2009}
% This article  illustrates a proposed integrated Bayesiannetwork  modeling  approach  using OOBNs within theiterative Bayesian network development cycle (IBNDC
% ^ This is a really comprehensive and well written definition of object orientated bayesian networks!
% 
%
%


%%
% integrated Bayesian networks
%
% The complex nature of environmental issues makes themideal case studies for the use of Bayesian networks as anintegration tool (Varis and Kuikka 1999; Uusitalo 2007;Barton et al. 2008), 

%%
% Self-organising systems.
%
% Sustainability is one such complex system. It comprises complex interacting factors and processes (e.g. in the context of the dairy study: farm, factory and market processes; and environmental, social and economic factors). It is self-organising; it does not require external intervention to thrive or deteriorate (e.g. an ecosystem can be sustainable through self-organisation). It can also exhibit emergent behaviour, since intervening in one part of the sustainability system can have unintended and quite extreme effects in seemingly unrelated parts of the system (Johnson and Mengersen, 2012).
%
% These systems are exactly what you think they are. It is described in physics, chemistry, social sciences and philosophy.
% 
% The wiki is quite informative.
%https://en.wikipedia.org/wiki/Self-organization
