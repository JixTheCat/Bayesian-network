%%%%%%%%%%%%%%%
% Paper setup %
%%%%%%%%%%%%%%%

% What is your topic?
% What is the context for your reserach?
% What do you want to achieve?
% Are there any relationships you want to explore?

%%%%%%%%%%%%%%%%
% Working title
%
\title{Measuring sustainability with Australian vine growers using a Bayesian Network}

% Description
% NB Avoid titles along the lines of: “Effects of ...”, “The role of ...”, etc.  Be specific about the effect and its significance so that your reader knows what is on offer.

% For example, rather than write a title like “The effect of factor X on astrophysical properties of green cheese.”, 

% be specific about the effect and write something more like ...

% “Factor X halves the lunar thermal diffusivity of  green cheese”.

% You will usually find it easier to write an effective title if you make your title a sentence.

% Notes
% N/a

%%%%%%%%%%%%%%%%%%%%
% Intended readers
%
% David Klassen (wine maker Taylors)
% Dpt Agriculture - EL worker
        % They would like to know for policy
% Lawson - asked about OOBNs
% Winegrower - Someone from the limestones coast. The guy from Robe was interested.

% Description:
% Name 4 or 5 potential readers - give their names and why they would be interested  (e.g. “Ichabod Crane, paleo-fudgologist interested in polygalactic fudginomiality”, not “assorted paleo-fudgologists”).   

% Your readers should be outside your institution.

% Notes
% Many viticulturalists I know of are on the panel. It makes it hard to get one in particular.

%%%%%%%%%%%%%%%%%%%%%
% Anticipated Journal
%
% OIV Congress 2024

% Descrption
% Ensure that all readers are likely to read the nominated journal  (e.g.  few non-researchers read refereed academic journals; politicians simply don’t read). 

% Notes
% Strangely this is less likely to be other researchers. Unless they are interested in Wine.

%%%%%%%%%%%%
% Question %
%%%%%%%%%%%%

% TODO: review the following questions:
% What are the key factors/indicators of sustainability in the australian winegrowing industry?
%  How can we measure and relate the factors that contribute to sustainability in the australian winegrowing industry?

% Things to consider
%
% Why is the knowledge important?
% What is the significance?
% How will the findings be utilised?
% What improvements may be derived from this result?
% Are the terms well defined?
% Is it doable? (cost & ethics)
%       Can you finish it in time?
% Do experts think your question is important/relevant/doable?

% What is the nature of your question?
%         who, what, where, when, why, how?

% You can have more than one question.

% What is the most important question your paper will pose?
%
% NB It is essential that your answer is framed as a direct question.  Your response must end with a question mark.

% Why is the question/issue/problem worth examining?
%
% Briefly outline the problem you are tackling and explain why the problem is important to knowledge in general.  “Nothing much is known” is not sufficient justification by itself.  You have to show why the gap in knowledge is important.  Expect to draw heavily on your reading of the literature in framing your answer but do not get into detail of  author and year.

%%%%%%%%%%
% Answer %
%%%%%%%%%%

% What is the answer to question
%
% 
% NB  You must give a direct answer to the question posed.
% Notes

% Will the findings be considered significant?

% How did you gather the evidence? 
%
% Briefly outline the methods you used to gather your evidence.

% What is the main evidence?
%
% Briefly outline the key results.  Focus on outcomes.

% What can you add to theory?
%  
% A research paper has to add to broader understanding. What will yours contribute?  Think about how your results and conclusions will change how people see the world.
%
% Many people have trouble with this section.   Do not recycle the results.  Focus on the conceptual models that explain why your results are as they are, or why they are different from what might have been expected.  Your contribution may be something new or it may be confirmation of something already known but in a slightly different context.
%
% Sometimes the contribution to theory is not a simple answer but a better understanding of the questions that ought to be asked in future.
%
% Again, expect to draw heavily on the literature in framing your answer, but cite the literature only sparingly here (you can go into full detail when you prepare your discussion).

% What can you add to practice?
%
% Superior research also has practical consequences.  What are the consequences of your work?  Think about how your results and conclusions might change what people do. Do not merely restate your results.

%%%%%%%%%%%%%%%
% Future work %
%%%%%%%%%%%%%%%

% What remains unresolved?
% 
% You may or may not have a lot to say here.  Some of it may be useful in your discussion.




%%%%%%%%%%%%%%%%%%%%%%%%%%%%%
%       Workshop ideas      %
%%%%%%%%%%%%%%%%%%%%%%%%%%%%%

% Ice breaker
% - who are you what are you involved in.
% Who knows who?
%

% Ask the expert team to nominate scenarios of interest then enter evidence into the BN to represent these scenarios.

%%%%%%%%%%%%%%%%%%%%%%%%%%%
% Notes for this article! %
%%%%%%%%%%%%%%%%%%%%%%%%%%%

% The creation of the model was done in the following stages (this is on page xxiv):

%  + The conceptual model
%  + key sustainability indicators
%  + Connection of key indicators to available measurable variables

% The identified measurable variables were:

% Econoomy (Farm to gate was used):
%  + Economics
        % + Debt
        % + Assets
        % + Investment
%  + profitability
        % + Interest
        % + Income
        % + Competitiveness
% %  + Work force
%         + Employment
%         + Productivity
%         + Management
%  + market
        % + Commodity prices
        % + Supply
        % + Risks
%  + physicals
        % + Efficiency
        % + Resource
        % + Inputs
%  + commodity prices
%  + Supply and risks

% Social (across farm, factory and market)
%  + employment
        % + Labour
        % + Training
        % + Management
%  + legal and ethics
        % + Discrimination
        % + Compliance (regulatory)
%  + community
        % + Culture
        % + Value
        % + Cohesion (participation/social responsibility)
%  + health and safety
        % + Animal welfare
        % + occupational
        % + Consumer
%  + product
        % + Nutrition
        % + marketing

% Environment
%  + Reource Efficiency
%         + Water
%         + Energy & emissions
%         + land and soil
%  + Waste
%         + Water Waste
%         + Solid Waste

% A high | low | medium score was used

%%%%%%%%%%%%%%

% The research for this project has identified that there is no uniform
% way to measure sustainability, no uniform agreement on indicators
% that could be used for measurement and even less agreement on
% which metrics should be used to quantify these indicators. This is
% best demonstrated that of the 72 triple bottom line frameworks, the
% indicator water occurred in 64 of these frameworks, although there
% appeared to be no standard metric identified to measure how water
% was measured. This is an endemic problem with sustainability
% measurement.

% The forces which drive the social, economic and environmental sectors, the decision-makers who have the power or influence to accept or reject change, cannot be ignored, if change is to be achieved (van Kerkhoff and Lebel, 2006).It may be that part of the problem with the progress of sustainable development in the past has been the traditional divide between those undertaking the systematic research and those on the ground required to take up and act on the results of such. The dissemination of research results has usually only occurred on completion, in a unidirectional, linear communication (van Kerkhoff and Lebel, 2006), whereas, as grounded researchers explain, the ownership of ideas and a willingness to implement these relies on a more collaborative approach from the outset, before research has even commenced. It requires acknowledging that experts exist in all spheres and need to be able to share information and debate, in order to come to agreement on issues, assessment tools, priorities for remediation and an optimal approach to solving problems (Corbin and Strauss, 2008; Johnson and Mengersen, 2009), e.g. improving sustainability.

% The sustainability Bayesian network model was constructed using
% open source software and has been delivered as a software package.
% A future project could focus on making it easier for a wider range of
% relevant stakeholders to engage with the model. For example a
% graphical user interface to the model could be developed and
% then web-enabled so that stakeholders can readily interact with the
% model and run scenarios.

% The concept of sustainability has been described as the most
% challenging policy concept ever developed (Spangenberg, 2004).
% There are more than 100 definitions of sustainability and sustainable
% development (Labuschagne et al., 2005). The most frequently cited
% definition comes from the pivotal 1987 World Commission on
% Environment and Development ‘Brundtland Report’, ‘development
% that meets the needs of the present without compromising the ability
% of future generations to meet their own needs’ (WCED, 1987).


% It is necessary to understand the impacts of each system on one another, for example, the impact of industries on climate change, diminishing resources, economic shifts and social change. In turn, it is necessary to understand the impact of variables such as these on an industry, its future viability, its surrounding communities and the economies they feed or depend on (Brundtland, 2007). Thus, a tool is required that can measure complexity within and between systems and model how changes in one element, positive or negative, might flow on to others. Only then can strategies for sustainability be developed with reduced risk of unintended negative consequences (Johnson and Mengersen, 2012).



%%%%%%%%%%%%%%%%%%%%%%%%%%%%

% references to check:


% https://www.sciencedirect.com/science/article/pii/S1462901122003872

%%%%
% A triple bottom line planning tool for measuring sustainability; A systems approach to sustinability using the Australian dairy industry as a case study.

% Capra, 1996; 
% Johnson and Mengersen, 2012)

% Murphy (2002)
%  A diagram of graphical models!

% The below references are from the previous OIV conference
%       Interestingly the theme was sustainability.

% 2023-3134: A DIGITAL TWIN APPLICATION FOR VINEYARDS SUSTAINABLE MANAGEMENT
% Vittorio Faluomi: Tecnovine srl, Italy, vittorio.faluomi@tecnovine.com

% 2023-3299: SUSTAINABLE PRODUCTION OF GRAPEVINE AND WINE IN TEXAS
% Amit Dhingra: Department of Horticultural Sciences, USA, amit.dhingra@ag.tamu.edu

% 2023-2958: DO SUSTAINABILITY CREDENTIALS NO LONGER SERVE AS A COMPETITIVE ADVANTAGE? - THE
% QUEST FOR COMPETITIVE PARITY WITH REGARD TO SUSTAINABILITY IN THE WINE INDUSTRY
% Barbara Richter: Hochschule Geisenheim University, Germany, barbara.richter@hs-gm.de

% 2023-3028: THE BLOCKCHAIN FOR THE SUSTAINABILITY OF DESIGNATIONS OF ORIGIN AND QUALITY AGRI-
% FOOD DISTRICTS: THE CASE OF VERMENTINO DI GALLURA D.O.C.G.
% Graziella Benedetto, Forleo Marina: University of Sassari, Italy, gbenedet@uniss.it

%  **************** This one looks super relavent
% 2023-3044: SUSTAINABILITY ASSESSMENT: TESTING AND VALIDATING A HIERARCHICAL FRAMEWORK IN THE
% PORTUGUESE WINE SECTOR CONTEXT.
% Ana Trigo, Ana Marta-Costa, Rui Fragoso: Centre for Transdisciplinary Development Studies (CETRAD), University of Trás-
% os-Montes e Alto Douro (UTAD)., Portugal, anatrigo@utad.pt

% 2023-3146: USER-GENERATED CONTENT AND RELEVANCE OF SUSTAINABILITY ATTRIBUTES FOR WINE
% CONSUMERS
% Miguel-Ángel Gómez-Borja, Inmaculada Carrasco, Juan-Sebastián Castillo: Universidad de Castilla-La Mancha, Spain,
% miguelangel.gborja@uclm.es

% 2023-2844: THE JOURNEY TO FARMLAND SOIL RESTORATION: UNDERSTANDING AND MEASURING
% PROGRESS TOWARDS SUSTAINABILITY
% Alberto Acedo: Biome Makers, Spain, acedo@biomemakers.com

% 2023-3239: ECOLOGICAL CRITERIA FOR A SUSTAINABILITY ASSESSMENT USING THE EXAMPLE OF THE ONLINE
% CERTIFICATION "NACHHALTIG AUSTRIA" (SUSTAINABLE AUSTRIA)
% Franz Rosner, Barbara Richter: Federal College and Research Institute for Viticulture and Pomology Klosterneuburg,
% Austria, franz.rosner@weinobst.at

% 2023-3084: THE IMPORTANCE OF RURAL EXTENSION AND ADVISORY SERVICES TO ACHIEVE A SUSTAINABLE
% VITICULTURE IN A CLIMATE CHANGE SCENARIO
% Ana Chambel: AVIPE, Portugal, ana.chambel@avipe.pt

% 2023-3220: EXPLORING BAYESIAN BELIEF NETWORK TO SUPPORT SUSTAINABLE VINEYARD IRRIGATION IN
% BRAZILIAN SEMI-ARID
% Luiz Gustavo Lovato: UNICAMP, Brazil, luizglovato@gmail.com

% PO-302
% 2023-2856: DATA DRIVEN APPROACH TO ENERGY EFFICIENCY IN WINERIES
% Gellio Ciotti, Marco Bietresato, Alessandro Zironi, Roberto Zironi, Rino Gubiani: Università degli Studi di Udine, Italy,
% gellio.ciotti@uniud.it



%%%%%%%%%%%%
% Definitions:

%%
% Iterative Bayesian network developement cycle
%
% \autocite{johnsonIntegratedBayesianNetwork2009}
% This article  illustrates a proposed integrated Bayesiannetwork  modeling  approach  using OOBNs within theiterative Bayesian network development cycle (IBNDC
% ^ This is a really comprehensive and well written definition of object orientated bayesian networks!
% 
%
%

%%
% integrated Bayesian networks
%
% The complex nature of environmental issues makes themideal case studies for the use of Bayesian networks as anintegration tool (Varis and Kuikka 1999; Uusitalo 2007;Barton et al. 2008), 
%
% Integrated models incorporate other types of tools like regression etc - it is literally what we are doing here. I would ask for a better definition from kerrie though. The term is used a lot but I cannot find an explicit definition.
%

%%
% Self-organising systems.
%
% Sustainability is one such complex system. It comprises complex interacting factors and processes (e.g. in the context of the dairy study: farm, factory and market processes; and environmental, social and economic factors). It is self-organising; it does not require external intervention to thrive or deteriorate (e.g. an ecosystem can be sustainable through self-organisation). It can also exhibit emergent behaviour, since intervening in one part of the sustainability system can have unintended and quite extreme effects in seemingly unrelated parts of the system (Johnson and Mengersen, 2012).
%
% These systems are exactly what you think they are. It is described in physics, chemistry, social sciences and philosophy.
% 
% The wiki is quite informative.
%https://en.wikipedia.org/wiki/Self-organization





% Initiatives to improve the environmental impacts of the Australian Wine industry have been ongoing since the 1990s; with recommendations to implement strategies for national programs that would help educate industry members in sustainable practices (Keith Jones, 2002). The first legislative initiative to support national programs was the Environmental Management Guidelines for Vineyards in Western Australia during 2002 (Nind, 2002). The first national program was the Australian Wine Industry Stewardship, beginning shortly after, during 2004 (WFA, 2009). This program was replaced multiple times during the 2000s, first by the Australian Wine Environmental Stewardship, and then by EnviroWine, finally becoming EntWine in 2009. The period in the 2000s was further influenced by several key researchers and program leaders, where concurrently multiple regions developed approaches to sustainable management tied to sustainability programs (SWA, 2023). EntWine was eventually acquired by the Australian Wine Research Institute and Australian Grape & Wine during 2015; which would lead to its redevelopment into Sustainable Winegrowing Australia (SWA) during 2019/2020 (SWA, 2023).
%             2.1.2. Previous Research and Motivation
% During the period of 2010 to 2023 data was collected voluntarily from wine industry members through the national sustainability programs, ultimately being in SWA possession as of 2023. The data set was used in 2020 to analyse sustainable practices, resulting in the development of a Resource Intensity Score (RIS); where the RIS was a summary indicator value of a vineyard’s energy use, emissions and water use (Nayak et al., 2020). The analysis then fit vineyards’ RIS values to 55 variables sourced from 23 separate wine regions during the period of 2015 to 2018 through the use of Stepwise Regression modelling. It was found that the top 10% gross margin vineyards had a lower operating cost and a lower RIS score than vineyards in the bottom 10% gross margin.
% The motivation for this thesis is to investigate the differences between the top 10% gross margin vineyards and those of the bottom 10%, hoping to uncover what the key differences were and to relate them to sustainable practices; as there is a known gap in the literature due to the difficulty to evaluate relationships using empirical evidence of both the environmental and economic impacts of sustainable practices in viticulture. This difficulty stems from data scarcity, industry complexity and transparency. While there have been attempts to fill these gaps, methodologies and indicators have lacked in ability to compare different regions and scales (Baiano, 2021).
%             2.1.3. Economic Pressures
% Historically strong demands for Australian wine have helped to create a thriving industry, however recently sharp reductions in exports to mainland China due to significant deposit tariffs have caused  a decline of 19% in Australian export value during the 2021-2022 financial year (see Figure 1) (Wine Australia, 2022). The pressure brought on by the drop in export value has been exacerbated by loss of tourism and labour due to the COVID-19 pandemic, global freight crisis, war in Europe and rising inflation (Wine Australia, 2021a, 2020). These pressures within the wine industry have trickled down to vineyards, where winemakers retain unwanted wine, creating an oversupply of grapes. Currently the proposed strategy to soften pressures proposed by Wine Australia includes initiatives focused on market diversification, and sustainability to build resilience in the coming years. 
% Figure 1:  The exports of Australian wine over time in Australian Dollars Free On Board, comparing exports between China and the rest of the world. This graphic is taken from the Wine Australia Annual Report of 2020-21(Wine Australia, 2022).
%             2.1.4. Environmental Pressures
% There are several environmental concerns that affect viticulture, including loss of soil quality, lack of rain, hail, disease, fire, and frost; with climate change exacerbating these issues. In 2020, 40,000 tonnes of grapes were lost across 18 different wine regions due to bush fires and smoke taint; the predicted incidence of wildfires is expected to increase (Canadell et al., 2021). In comparison to countrywide pressures such as drought, this damage made up only 3% of the total amount of grapes for that year; although acknowledged as a considerable loss on an individual basis, it was deemed to be only a minor national concern by Wine Australia when compared to other environmental pressures such as drought (Wine Australia, 2020). 



%
%This analysis addresses the knowledge gap regarding the effectiveness of regional level strategies employed in the wine industry and their relation to grape quality.
% What the hell is a regional strategy
% And how do you define, or know what trategy each region is using?
%
% Through the use of classification trees this study aims to highlight the key differences in sustainable practices at a regional level and how these practices relate to the different grades of grape quality.

% This does not really address what the methods are - aside from classisifcation trees in the most vague sense.
% What are the results briefly? why should someone read this article?
% What is interesting - what will be discussed in the article with regard to these results? What makes me want to read the article?
% And what is the take away point of the article? If I cant be bothered to read this article, what piece of knowledge can you leave mewiwth that I could recommend the article to someone else with?


\section{Methods}
\citep{mayfieldDesigningExpertledBayesian2023} -> this reference also uses a similar method which we can cite alongside everything else.
% These are transcribed notes from Korb & Nicholson:

\citep{korbBayesianArtificialIntelligence2011} Korb & Nicholson describe three major issues in the creation of Bayesian Networks though knowledge Engineering:
- Generating Arc Structure
- Parametarisation
- dealing with utilities

Korb and Nicholson also describe 16 common problems that occur during the construction of Bayesian. Korb and Nicholson suggest methods for overcoming this similarly to Pitchforth:
\subsection{The process}
\subsubsection{Parameterizing before evaluating the structure}
\begin{enumerate}
        \item Are the nodes in the BN the right ones?
        \item Are the state spaces (values) for each node correct?
        \item Are the arcs right?
\end{enumerate}
If the answer to any of the above questions is no; then the structure will not be sufficiently defined for the parametarisation of the CPT. In this paper we explicitly defined the structure of the BN prior to CPT parametarisation. This involved:
\begin{enumerate}
        \item exposing the panel to a strawman model in an intial workshop
        \item Describing the model and the process invovled in defining the structure of a BN
        \item The expert panel reviewing possible nodes and arcs, defining their own structures prior to a second workshop (with correspondence between expert panel members prior to the next workshop)
        \item A second workshop to agree upon the final structure of the model
\end{enumerate}
\subsubsection{Trying to build the full model all at once}
Korb & Nicholson say that it is nearly impossible to build a complex BN all in one go. They recommend:
\begin{enumerate}
        \item dividing models into subcomponents
        \item breaking down the problem into a single target variable
        \item reducing the scope of the problem
        \item Setting a single time frame for the problem
\end{enumerate}
The strawman was presented as three subcomponents that pertained to the triple bottom line's environment, economy and social aspects.
Within the first workshop it was agreed on by the expert panel to reduce the scope from a triple bottom line to only reviewing the environmental impacts of vineyards. This also reduced the BN to a single terminal node. It was further decided to create the BN as a single season to limit its complexity.

\subsection{The problem}
\subsubsection{Not understanding the problem context, complexity without value.}
It is crucial to gain a clear understanding of the problem and its context. This can be seen as answering the following questions:
\begin{enumerate}
        \item what do you want to reason about?
        \item What dont you know?
        \item What information do you have?
        \item What do you know?
These questions are well addressed in prior research. The creation of baseline models showed strong linear relationships between many variables. The creation of XGBoosted trees uncovered the ability to well inform any particular variable but with no real causal structure or understandability. here we seek to pursue causality and link these variables together in a way that is more aligned with a decision support structure than purely as a predictive exercise.
\subsubsection{Complexity without value}
A very common impulse, when something isk nown about thep roblem, is to want to put it in the model. But including everything known, just because it is known, simply adds complexity to the model without adding any value (and in fact often reduces value). Instead, the knowledge engineer must focus on the problem:
"Which of the known variables are most relevant to the problem?"
- This issue in its the simplest form has attempted to be overcome through trying to mediate the workshop and inform experts of the potential probelsm that will come with a cumbersome model.
\end{enumerate}
\subsection{Structure - Nodes}
The list below is each of the problems, which can be addressed together in some fashion as the approaches of solving them overlap.
\begin{enumerate}
        \item Getting the node values wrong
        \item Node values aren't exhaustive
        \item Node values aren't mutually exclusive
        \item Incorrect modelling of mutually exclusive outcomes
        \item Trying to model fuzzy categories
        \item Confusing state and Probability
        \item Confusion about what the node represents
\end{enumerate}
Two major strategies were employed to overcome these problems. Tying nodes explicitly to a data source that corroberated the definition and state in some manner as the node. For node s that did not have data set they were sought to be defined with an explicit explanation that experts agreed upon or had some external explanation such as legislation.
\subsection{Structure Arcs}
\subsubsection{Getting the arc directions wrong}
Arc directions are commonly wrong for the following reasons:
\begin{enumerate}
        \item Modelling reasoning rather than causation
        \item Inverting cause and effect
        \item Missing variables
\end{enumerate}
To note: Korb and Nicholson do not actually suggest a method for overcoming these problems. There is an inherint risk in creating a model and not having knowledge of a missing variable or of a causation. It is important to acknowledge that there is always a possibility that causation is mixed up with reasoning, and that not all variables are always known.
\subsubsection{Too many parents}
Sometimes it is unavoidable that a node is affected by many factors, or other nodes. There are some methods to help overcome this such as divorcing, in which a node is created to capture the cumulative affects of several parents on a child. This technique was employed in the creation of the structure of the BN with the expert panel.
\subsection{Parameters}
\subsubsection{Experts estimates of probabilities are biased}
\begin{enumerate}
        \item Overconfidence: The attribution of higher than justifiable probabilities to nodes/events.
        \indent The tendency to attribute higher than justifiable probabilities (\citep{lichtensteinCalibrationProbabilitiesState1982} 
        \item Anchoring: The tendency towards higher/lower based off of an arbitrary point of comparison. "Anchoring" people at a higher or lower than normal prediction because of some assumption or example  \citep{kahnemanPsychologyPrediction1973}.
        \item Availability: The assessing of an event as more probable than is justifiable becaues it is more memorable or noticeable \citep{tverskyJudgmentUncertaintyHeuristics1974}.

\end{enumerate}
We have attempted to overcome this with two methods. A diverse and varried expert panel. Those participanting have experience in multiple vineyards as managers and other duties as well as the panel itself having varied manager/owners from different regions and scale. With some experts being more specialised in specific areas such as agro-chemicals.
\subsubsection{Inconsistent probabilities}
The collection of probabilities for when the CPT is large can result in inconsistent probabilities. The collection of probabilities at different times or if probabilities are supplied by different experts can result in inconsistent probabilities as well.
The use of desciptors such as very high or low can be of assistance \cite{vandergaag2013elicit}. In this study we overcame this through the use of worded response such as high and low. We then assigned probabilities to those broader cases. We had every expert agree upon each of the probabilities so that they could argue for or against any particular bias and so that the probabilities would even out to a more broader applicability, across all of the experts experiences.
% \item Inconsistent "filling in" of large CPTs:
-       -       - Being dead certain.
\subsubsection{Incoherent probabilities (not summing to 1)}
We used software that automatically detected incoherent probabilities when filling in the CPT values to overocme this problem.
\subsubsection{Being dead certain}
It is important that we limit over confidence (A quote form mark twain was used in the korb and Nicholson). 
Our best defence against this was to use multiple experts with different experiences and to challenge them when something was certain. To ask why and to check in about their thoughts.

\citep{morganUncertaintyGuideDealing1990}
Two attributes of good elicitation process are:
\begin{enumerate}
        \item The expert should be apprised of what is known about the process, especially the nearly universal tendency to overcomefidence and other forms of bias. In order to avoid some problems, values should be elicited in random order and the expert not given feedback on how the different values fit together until a complet set has been elicited.
        \item The elicitation process is not simply one of requesting and recording numbers but also one of the refining the definitions of variables and terms to be used in the model. What values are elicited depends directly upon the interpretation of terms and these should be made as explicit as possible and recorded during the elicitation. This is part of the process management described earlier.
\end{enumerate}
- 

\citep{Boneh2021} Boneh, T. (2010). Ontology and Bayesian decision networks for supporting the meterological forecasting process. Ph. D thesis, Clayton School of infomration technology, Monash university
Boneh 2010 describe Knowledge Engineering with Bayesian Networks (KEBN) as occurring in a 4 stage life cycle:
- Bayesian Network Structure
- Probability Parameters
- Decision Structure
- Utilities Preference

Pitchforth \citep{pitchforthProposedValidationFramework2013} agrees with this style of structure naming these steps as:
- Structure
- Discretisation
- Parametarisation
- Model behaviour
- Validation

Korb & Nicholson have a similar view to Pitchforth. utilising a waterfall style developemental approach which include 5 stages:
- Building the BN
-       - Structure
-       - Parameters
-       - Preferences
- Validation
-       - Sensitivity Analysis
-       - Accuracy testing
- Field testing
-       - Alpha/Beta testing
-       - Acceptance testing
- Industrial use
-       - Collection of Statistics
- Refinement
-       - Updating procedures
-       - Regression testing



Where discretisation and validation are additional steps to assign intervals to continuous factors (discretisation) and to ensure the models appropriateness (validation)

With validation incorporating the expert scrutiny of the models begaviour for scenarios, as well as comparing this to both data and other literature.



\citep{korbBayesianArtificialIntelligence2011}













