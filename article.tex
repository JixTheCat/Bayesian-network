%%%%%%%%%%%%%%%
% Paper setup %
%%%%%%%%%%%%%%%

% What is your topic?
% What is the context for your reserach?
% What do you want to achieve?
% Are there any relationships you want to explore?

%%%%%%%%%%%%%%%%
% Working title
%
\title{Measuring sustainability with Australian vine growers using a Bayesian Network}

% Description
% NB Avoid titles along the lines of: “Effects of ...”, “The role of ...”, etc.  Be specific about the effect and its significance so that your reader knows what is on offer.

% For example, rather than write a title like “The effect of factor X on astrophysical properties of green cheese.”, 

% be specific about the effect and write something more like ...

% “Factor X halves the lunar thermal diffusivity of  green cheese”.

% You will usually find it easier to write an effective title if you make your title a sentence.

% Notes
% N/a

%%%%%%%%%%%%%%%%%%%%
% Intended readers
%
% David Klassen (wine maker Taylors)
% Dpt Agriculture - EL worker
        % They would like to know for policy
% Lawson - asked about OOBNs
% Winegrower - Someone from the limestones coast. The guy from Robe was interested.

% Description:
% Name 4 or 5 potential readers - give their names and why they would be interested  (e.g. “Ichabod Crane, paleo-fudgologist interested in polygalactic fudginomiality”, not “assorted paleo-fudgologists”).   

% Your readers should be outside your institution.

% Notes
% Many viticulturalists I know of are on the panel. It makes it hard to get one in particular.

%%%%%%%%%%%%%%%%%%%%%
% Anticipated Journal
%
% OIV Congress 2024

% Descrption
% Ensure that all readers are likely to read the nominated journal  (e.g.  few non-researchers read refereed academic journals; politicians simply don’t read). 

% Notes
% Strangely this is less likely to be other researchers. Unless they are interested in Wine.

%%%%%%%%%%%%
% Question %
%%%%%%%%%%%%

% Things to consider
%
% Why is the knowledge important?
% What is the significance?
% How will the findings be utilised?
% What improvements may be derived from this result?
% Are the terms well defined?
% Is it doable? (cost & ethics)
%       Can you finish it in time?
% Do experts think your question is important/relevant/doable?

% What is the nature of your question?
%         who, what, where, when, why, how?

% You can have more than one question.

% What is the most important question your paper will pose?
%
% NB It is essential that your answer is framed as a direct question.  Your response must end with a question mark.

% Why is the question/issue/problem worth examining?
%
% Briefly outline the problem you are tackling and explain why the problem is important to knowledge in general.  “Nothing much is known” is not sufficient justification by itself.  You have to show why the gap in knowledge is important.  Expect to draw heavily on your reading of the literature in framing your answer but do not get into detail of  author and year.

%%%%%%%%%%
% Answer %
%%%%%%%%%%

% What is the answer to question
%
% 
% NB  You must give a direct answer to the question posed.
% Notes

% Will the findings be considered significant?

% How did you gather the evidence? 
%
% Briefly outline the methods you used to gather your evidence.

% What is the main evidence?
%
% Briefly outline the key results.  Focus on outcomes.

% What can you add to theory?
%  
% A research paper has to add to broader understanding. What will yours contribute?  Think about how your results and conclusions will change how people see the world.
%
% Many people have trouble with this section.   Do not recycle the results.  Focus on the conceptual models that explain why your results are as they are, or why they are different from what might have been expected.  Your contribution may be something new or it may be confirmation of something already known but in a slightly different context.
%
% Sometimes the contribution to theory is not a simple answer but a better understanding of the questions that ought to be asked in future.
%
% Again, expect to draw heavily on the literature in framing your answer, but cite the literature only sparingly here (you can go into full detail when you prepare your discussion).

% What can you add to practice?
%
% Superior research also has practical consequences.  What are the consequences of your work?  Think about how your results and conclusions might change what people do. Do not merely restate your results.

%%%%%%%%%%%%%%%
% Future work %
%%%%%%%%%%%%%%%

% What remains unresolved?
% 
% You may or may not have a lot to say here.  Some of it may be useful in your discussion.




%%%%%%%%%%%%%%%%%%%%%%%%%%%%%
%       Workshop ideas      %
%%%%%%%%%%%%%%%%%%%%%%%%%%%%%

% Ice breaker
% - who are you what are you involved in.
% Who knows who?
%

% Ask the expert team to nominate scenarios of interest then enter evidence into the BN to represent these scenarios.

%%%%%%%%%%%%%%%%%%%%%%%%%%%
% Notes for this article! %
%%%%%%%%%%%%%%%%%%%%%%%%%%%

% The creation of the model was done in the following stages (this is on page xxiv):

%  + The conceptual model
%  + key sustainability indicators
%  + Connection of key indicators to available measurable variables

% The identified measurable variables were:

% Econoomy (Farm to gate was used):
%  + Economics
        % + Debt
        % + Assets
        % + Investment
%  + profitability
        % + Interest
        % + Income
        % + Competitiveness
% %  + Work force
%         + Employment
%         + Productivity
%         + Management
%  + market
        % + Commodity prices
        % + Supply
        % + Risks
%  + physicals
        % + Efficiency
        % + Resource
        % + Inputs
%  + commodity prices
%  + Supply and risks

% Social (across farm, factory and market)
%  + employment
        % + Labour
        % + Training
        % + Management
%  + legal and ethics
        % + Discrimination
        % + Compliance (regulatory)
%  + community
        % + Culture
        % + Value
        % + Cohesion (participation/social responsibility)
%  + health and safety
        % + Animal welfare
        % + occupational
        % + Consumer
%  + product
        % + Nutrition
        % + marketing

% Environment
%  + Reource Efficiency
%         + Water
%         + Energy & emissions
%         + land and soil
%  + Waste
%         + Water Waste
%         + Solid Waste

% A high | low | medium score was used

%%%%%%%%%%%%%%

% The research for this project has identified that there is no uniform
% way to measure sustainability, no uniform agreement on indicators
% that could be used for measurement and even less agreement on
% which metrics should be used to quantify these indicators. This is
% best demonstrated that of the 72 triple bottom line frameworks, the
% indicator water occurred in 64 of these frameworks, although there
% appeared to be no standard metric identified to measure how water
% was measured. This is an endemic problem with sustainability
% measurement.

% The forces which drive the social, economic and environmental sectors, the decision-makers who have the power or influence to accept or reject change, cannot be ignored, if change is to be achieved (van Kerkhoff and Lebel, 2006).It may be that part of the problem with the progress of sustainable development in the past has been the traditional divide between those undertaking the systematic research and those on the ground required to take up and act on the results of such. The dissemination of research results has usually only occurred on completion, in a unidirectional, linear communication (van Kerkhoff and Lebel, 2006), whereas, as grounded researchers explain, the ownership of ideas and a willingness to implement these relies on a more collaborative approach from the outset, before research has even commenced. It requires acknowledging that experts exist in all spheres and need to be able to share information and debate, in order to come to agreement on issues, assessment tools, priorities for remediation and an optimal approach to solving problems (Corbin and Strauss, 2008; Johnson and Mengersen, 2009), e.g. improving sustainability.

% The sustainability Bayesian network model was constructed using
% open source software and has been delivered as a software package.
% A future project could focus on making it easier for a wider range of
% relevant stakeholders to engage with the model. For example a
% graphical user interface to the model could be developed and
% then web-enabled so that stakeholders can readily interact with the
% model and run scenarios.

% The concept of sustainability has been described as the most
% challenging policy concept ever developed (Spangenberg, 2004).
% There are more than 100 definitions of sustainability and sustainable
% development (Labuschagne et al., 2005). The most frequently cited
% definition comes from the pivotal 1987 World Commission on
% Environment and Development ‘Brundtland Report’, ‘development
% that meets the needs of the present without compromising the ability
% of future generations to meet their own needs’ (WCED, 1987).


% It is necessary to understand the impacts of each system on one another, for example, the impact of industries on climate change, diminishing resources, economic shifts and social change. In turn, it is necessary to understand the impact of variables such as these on an industry, its future viability, its surrounding communities and the economies they feed or depend on (Brundtland, 2007). Thus, a tool is required that can measure complexity within and between systems and model how changes in one element, positive or negative, might flow on to others. Only then can strategies for sustainability be developed with reduced risk of unintended negative consequences (Johnson and Mengersen, 2012).



%%%%%%%%%%%%%%%%%%%%%%%%%%%%

% references to check:


% https://www.sciencedirect.com/science/article/pii/S1462901122003872

%%%%
% A triple bottom line planning tool for measuring sustainability; A systems approach to sustinability using the Australian dairy industry as a case study.

% Capra, 1996; 
% Johnson and Mengersen, 2012)

% Murphy (2002)
%  A diagram of graphical models!

% The below references are from the previous OIV conference
%       Interestingly the theme was sustainability.

% 2023-3134: A DIGITAL TWIN APPLICATION FOR VINEYARDS SUSTAINABLE MANAGEMENT
% Vittorio Faluomi: Tecnovine srl, Italy, vittorio.faluomi@tecnovine.com

% 2023-3299: SUSTAINABLE PRODUCTION OF GRAPEVINE AND WINE IN TEXAS
% Amit Dhingra: Department of Horticultural Sciences, USA, amit.dhingra@ag.tamu.edu

% 2023-2958: DO SUSTAINABILITY CREDENTIALS NO LONGER SERVE AS A COMPETITIVE ADVANTAGE? - THE
% QUEST FOR COMPETITIVE PARITY WITH REGARD TO SUSTAINABILITY IN THE WINE INDUSTRY
% Barbara Richter: Hochschule Geisenheim University, Germany, barbara.richter@hs-gm.de

% 2023-3028: THE BLOCKCHAIN FOR THE SUSTAINABILITY OF DESIGNATIONS OF ORIGIN AND QUALITY AGRI-
% FOOD DISTRICTS: THE CASE OF VERMENTINO DI GALLURA D.O.C.G.
% Graziella Benedetto, Forleo Marina: University of Sassari, Italy, gbenedet@uniss.it

%  **************** This one looks super relavent
% 2023-3044: SUSTAINABILITY ASSESSMENT: TESTING AND VALIDATING A HIERARCHICAL FRAMEWORK IN THE
% PORTUGUESE WINE SECTOR CONTEXT.
% Ana Trigo, Ana Marta-Costa, Rui Fragoso: Centre for Transdisciplinary Development Studies (CETRAD), University of Trás-
% os-Montes e Alto Douro (UTAD)., Portugal, anatrigo@utad.pt

% 2023-3146: USER-GENERATED CONTENT AND RELEVANCE OF SUSTAINABILITY ATTRIBUTES FOR WINE
% CONSUMERS
% Miguel-Ángel Gómez-Borja, Inmaculada Carrasco, Juan-Sebastián Castillo: Universidad de Castilla-La Mancha, Spain,
% miguelangel.gborja@uclm.es

% 2023-2844: THE JOURNEY TO FARMLAND SOIL RESTORATION: UNDERSTANDING AND MEASURING
% PROGRESS TOWARDS SUSTAINABILITY
% Alberto Acedo: Biome Makers, Spain, acedo@biomemakers.com

% 2023-3239: ECOLOGICAL CRITERIA FOR A SUSTAINABILITY ASSESSMENT USING THE EXAMPLE OF THE ONLINE
% CERTIFICATION "NACHHALTIG AUSTRIA" (SUSTAINABLE AUSTRIA)
% Franz Rosner, Barbara Richter: Federal College and Research Institute for Viticulture and Pomology Klosterneuburg,
% Austria, franz.rosner@weinobst.at

% 2023-3084: THE IMPORTANCE OF RURAL EXTENSION AND ADVISORY SERVICES TO ACHIEVE A SUSTAINABLE
% VITICULTURE IN A CLIMATE CHANGE SCENARIO
% Ana Chambel: AVIPE, Portugal, ana.chambel@avipe.pt

% 2023-3220: EXPLORING BAYESIAN BELIEF NETWORK TO SUPPORT SUSTAINABLE VINEYARD IRRIGATION IN
% BRAZILIAN SEMI-ARID
% Luiz Gustavo Lovato: UNICAMP, Brazil, luizglovato@gmail.com

% PO-302
% 2023-2856: DATA DRIVEN APPROACH TO ENERGY EFFICIENCY IN WINERIES
% Gellio Ciotti, Marco Bietresato, Alessandro Zironi, Roberto Zironi, Rino Gubiani: Università degli Studi di Udine, Italy,
% gellio.ciotti@uniud.it



%%%%%%%%%%%%
% Definitions:

%%
% Iterative Bayesian network developement cycle
%
% \autocite{johnsonIntegratedBayesianNetwork2009}
% This article  illustrates a proposed integrated Bayesiannetwork  modeling  approach  using OOBNs within theiterative Bayesian network development cycle (IBNDC
% ^ This is a really comprehensive and well written definition of object orientated bayesian networks!
% 
%
%

%%
% integrated Bayesian networks
%
% The complex nature of environmental issues makes themideal case studies for the use of Bayesian networks as anintegration tool (Varis and Kuikka 1999; Uusitalo 2007;Barton et al. 2008), 
%
% Integrated models incorporate other types of tools like regression etc - it is literally what we are doing here. I would ask for a better definition from kerrie though. The term is used a lot but I cannot find an explicit definition.
%

%%
% Self-organising systems.
%
% Sustainability is one such complex system. It comprises complex interacting factors and processes (e.g. in the context of the dairy study: farm, factory and market processes; and environmental, social and economic factors). It is self-organising; it does not require external intervention to thrive or deteriorate (e.g. an ecosystem can be sustainable through self-organisation). It can also exhibit emergent behaviour, since intervening in one part of the sustainability system can have unintended and quite extreme effects in seemingly unrelated parts of the system (Johnson and Mengersen, 2012).
%
% These systems are exactly what you think they are. It is described in physics, chemistry, social sciences and philosophy.
% 
% The wiki is quite informative.
%https://en.wikipedia.org/wiki/Self-organization





